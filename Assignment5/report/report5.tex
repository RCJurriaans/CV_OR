\documentclass[a4paper,10pt]{article}
\usepackage[utf8x]{inputenc}

%opening
\title{Epipolar Geometry}
\author{Robrecht Jurriaans, Taco Cohen}

\begin{document}

\maketitle

\section{Estimating the Fundamental Matrix}

In order to estimate the 3D geometry of the scene from stereo images, we need to know the epipolar geometry of the scene.
The easiest treatment of epipolar geometry can be made when the image planes are parallel, but this is generally not the case for our input images.
For this reason, we first estimate the fundamental matrix $F$, that maps one image to another by a projective transformation.

To estimate $F$, we need correspondences, for which we will use points.
We find corresponding points in two images by matching Harris/Hessian Scale+Affine invariant descriptors.
Next, we center the two point-sets ${p_i}$ and ${p'_i}$ (one for each image), and scale the points so that the average distance to the mean is $\sqrt{2}$.
The matrices that perform this translation-scaling we call $T$ and $T'$.
As shown by Hartley CITE, this transformation improves the numerical stability of the following estimation procedure.

The Fundamental matrix is defined by the constraints
\begin{equation}
\mathbf{p_i'} F \mathbf{p_i} = 0
\end{equation}
which is linear in the parameters of $F$.
We form the matrix $A$ containing the coefficients of the linear equations in its rows, and solve $A \mathbf{f} = 0$ using SVD, where $\mathbf{f}$ is the matrix $F$ reshaped to a vector.




\end{document}
