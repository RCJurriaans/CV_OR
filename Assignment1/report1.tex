\documentclass[a4paper,10pt]{article}
\usepackage[utf8x]{inputenc}

\usepackage{amsmath}

%opening
\title{Computer Vision Assignment 1: Filtering}
\author{Robrecht Jurriaans (5887380), Taco Cohen (6394590)}

\begin{document}

\maketitle

\section{Gaussian Filters}

\subsection{1D Gaussian Filter}
We implemented the 1D Gaussian in \verb+gaussian.m+.
We made sure the kernel size is about $3*\sigma$ and is always odd.
For this purpose we used the formula $2 * \lfloor 1.5 * \sigma \rfloor + 1$.

Because the filter has a finite size, the sum of the filter values will not be one in a naive implementation.
For this reason, we must normalize the kernel after calculating the values of the Gaussian at each kernel entry.
To save computation, we used the following equality:

\begin{align*}
 \frac{G_{\sigma}(x)}{\sum_{x'=-h}^h G_{\sigma}(x')} &= \frac{ \frac{1}{\sigma \sqrt{2 \pi}} \exp(-\frac{x^2}{2 \sigma^2})}
       {\sum_{x'=-h}^h \frac{1}{\sigma \sqrt{2 \pi}} \exp(-\frac{x'^2}{2 \sigma^2})} \\
&= \frac{ \exp(-\frac{x^2}{2 \sigma^2})}
       {\sum_{x'=-h}^h \exp(-\frac{x'^2}{2 \sigma^2})}
\end{align*}

That is, we leave out the normalization of each index, because it falls out in the normalization of the entire kernel anyway.

\subsection{}

\end{document}
